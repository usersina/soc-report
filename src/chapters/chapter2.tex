\chapter{State of the art}
\minitoc
\newpage

\setcounter{secnumdepth}{0} % Set the section counter to 0 so next section is not counted in toc
% ----------------------- Introduction ----------------------- %
\section{Introduction}
This chapter will present and study the state of the art in \glsxtrshort{soc}.
It will delve into various aspects of SoCs, including their evolution, core components, emerging trends, and comparative analysis.
Each section will contribute to a comprehensive understanding of the significance of SoCs in modern cybersecurity practices.

\setcounter{secnumdepth}{2} % Resume counting the sections for the toc with a depth of 2 (Sections and sub-sections)
% ----------------------------------- SECTIONS (v) ----------------------------------- %
% ----------------------- Definition of a SoC ----------------------- %
\section{Definition of a \glsxtrshort{soc}}
A \glsxtrshort{soc} is a centralized facility within an organization tasked with monitoring, detecting, analyzing, and responding to cybersecurity threats and incidents in real-time.
SoCs are equipped with advanced tools, technologies, and trained personnel to ensure the continuous protection of organizational assets and sensitive information.
By consolidating security operations into a single location, SoCs enable efficient coordination and collaboration among cybersecurity teams, thereby enhancing the organization's ability to defend against evolving cyber threats.

% ----------------------- Evolution of SoCs ----------------------- %
\section{Evolution of SoCs}
The evolution of SoCs can be traced back to the early days of computing when organizations first began to recognize the need for centralized security monitoring and incident response capabilities.
Over time, SoCs have undergone significant transformation, driven by advancements in technology, changes in the threat landscape, and regulatory requirements.
Key milestones in the evolution of SoCs include the development of Security Information and Event Management (SIEM) systems, the adoption of threat intelligence sharing frameworks, and the shift towards proactive threat hunting approaches.
Today, SoCs have evolved into sophisticated command centers equipped with advanced tools and technologies to combat cyber threats proactively.

\subsection{Core Components of a SoC}
At the heart of every SoC are several core components that collectively enable effective cybersecurity operations.
These include continuous monitoring capabilities to detect and analyze security events in real-time, threat detection mechanisms such as Intrusion Detection Systems (IDS) and Endpoint Detection and Response (EDR) solutions, incident response protocols for swift and coordinated action in the event of a security incident, and integration with threat intelligence sources to stay abreast of the latest threats and vulnerabilities.
Additionally, SoCs often incorporate automation and orchestration tools to streamline routine tasks and improve operational efficiency.

\subsection{Emerging Trends in SoCs}
In recent years, several emerging trends have reshaped the landscape of SoCs, offering new opportunities and challenges for cybersecurity practitioners.
One such trend is the adoption of cloud-based SoCs, which leverage the scalability and flexibility of cloud infrastructure to enhance agility and reduce operational costs.
Another significant trend is the integration of artificial intelligence (AI) and machine learning (ML) technologies into SoC operations, enabling more advanced threat detection capabilities and automated response actions.
Furthermore, there is a growing emphasis on threat intelligence sharing and collaboration among SoCs, as organizations recognize the value of collective defense against cyber threats.

% ----------------------- Comparative Analysis ----------------------- %
\section{Comparative Analysis}
When considering the implementation of a SoC, organizations must weigh various factors and considerations to determine the most suitable approach for their specific needs.

\subsection{Comparing: In-house SoC vs MSSPs}
One important consideration is whether to build and operate an in-house SoC or outsource these capabilities to a managed security service provider (MSSP).
\begin{table}[H]
    \renewcommand{\arraystretch}{1.5}%
    \caption{Comparison of In-house SoCs and Managed Security Service Providers (MSSPs)}
    \centering
    \medskip
    \begin{tabularx}{1\textwidth} {
            | >{\hsize=.7\hsize\linewidth=\hsize\centering\arraybackslash}X
            | >{\hsize=1.15\hsize\linewidth=\hsize\justifying\arraybackslash}X
            | >{\hsize=1.15\hsize\linewidth=\hsize\justifying\arraybackslash}X |}
        \hline
        \rowcolor{primary}                 & \textbf{In-house SoCs}                                                                                                                                                                           & \textbf{MSSPs}                                                                                                                                                  \\
        \hline
        \textbf{Control and Customization} & Offer greater control and customization over security operations and infrastructure, allowing organizations to tailor SoC capabilities to their specific needs.                                  & Provide access to specialized skills and technologies, enabling organizations to leverage external expertise for comprehensive security solutions.              \\
        \hline
        \textbf{Resource and Expertise}    & Require significant investment in resources and expertise for setup, maintenance, and operation of in-house SoCs.                                                                                & Offload the operational burden to a third-party MSSP, eliminating the need for internal resources and expertise dedicated to SoC operations.                    \\
        \hline
        \textbf{Architectural Approach}    & Organizations must decide on the architectural approach for their SoC, choosing between centralized and distributed architectures based on factors such as scalability, resilience, and latency. & MSSPs typically offer centralized SoC architectures, simplifying management and ensuring consistent security policies across the organization's infrastructure. \\
        \hline
        \textbf{Tools and Platforms}       & Organizations have the flexibility to choose between commercial and open-source tools and platforms, based on factors such as functionality, cost, and support.                                  & MSSPs may have preferred tools and platforms, offering standardized solutions with fixed feature sets and pricing models.                                       \\
        \hline
    \end{tabularx}
\end{table}

In the end, both in-house SoCs and MSSPs have their advantages and challenges.
In our case, we will be implementing an in-house SoC since it is easier to implement in the context of our project.
% ----------------------------------- SECTIONS (^) ----------------------------------- %

\setcounter{secnumdepth}{0} % Set the section counter to 0 so next section is not counted in toc
% ----------------------- Conclusion ----------------------- %
\section{Conclusion}
In conclusion, this chapter provided an overview of \glsxtrshort{soc}, covering their definition, evolution, core components, emerging trends, and a comparative analysis of implementation approaches. Overall, it emphasized the critical role of SoCs in modern cybersecurity practices.
